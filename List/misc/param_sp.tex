%================================================================
\typeout{MK Styleparts}
%================================================================
\usepackage{mksty} 
\usepackage{longtable}
%================================================================
\typeout{Springer Package}
%================================================================
%\usepackage{springer}
%================================================================
\typeout{Picktex Package}
%================================================================
%\input prepicte
%\input pictex
%\input postpict
%================================================================
\typeout{Language and encoding and size}
%================================================================
\usepackage[english]{babel}%
\selectlanguage{\english}
\usepackage[T1]{fontenc}%
\usepackage[ansinew]{inputenc}%
\usepackage{color}%
%\usepackage{a4}%
%================================================================
% MK fonts
%================================================================
\usepackage{amsfonts}
\usepackage{amssymb}
%================================================================
\typeout{Latex Support}
%================================================================
\usepackage{makeidx}  
\usepackage{multicol}
\usepackage{lscape}
%================================================================
\typeout{PDF Latex Support}
%================================================================
 \usepackage[pdftex,
            colorlinks=true,            % Schrift von Links in Farbe (true), sonst mit Rahmen (false)
            bookmarksnumbered=true,     % Lesezeichen im pdf mit Nummerierung
            bookmarksopen=true,         % �ffnet die Lesezeichen vom pdf beim Start
            bookmarksopenlevel=0,       % Default ist maxdim
            pdfstartview=FitH,          % startet mit Seitenbreite
            linkcolor=blue,             % Standard 'red'
            citecolor=blue,             % Standard 'green'
            urlcolor=blue,              % Standard 'cyan'
            filecolor=blue,             %
            pdfauthor = {\Authors},
            pdftitle ={\DocTitle -- Version \VersionNumber},
%            bookmarks = true,
            plainpages=false,pdfpagelabels]{hyperref} %
\usepackage[pdftex]{graphicx}

\hypersetup{colorlinks,%
citecolor=black,%
filecolor=black,%
linkcolor=black,%
urlcolor=black,%
pdftex}

%================================================================
\typeout{Customising Sizes}
%================================================================

%================================================================
\typeout{Customising Sizes}
%================================================================

\setlength{\paperwidth}{210mm}
\setlength{\paperheight}{297mm}%
%\setlength{\paperwidth}{153mm}
%\setlength{\paperheight}{237mm}%
\setlength{\textwidth}{117mm}
\setlength{\textheight}{191mm}
\setlength{\topmargin}{0cm}

\parindent 0mm
\parskip 1.2ex plus 0.2ex minus 0.2ex
\parskip \baselinestretch\parskip

\parindent 0mm
\parskip 1.2ex plus 0.2ex minus 0.2ex
\parskip \baselinestretch\parskip


%\usepackage{enumerate}%
%\usepackage{amsfonts,amsmath,amssymb,amsthm} %% txfonts hinzugef\"{u}gt 2.12.2002
%\usepackage[latin1]{inputenc}
%\usepackage{amssymbb} %% txfonts hinzugef\"{u}gt 2.12.2002
%\arraycolsep2pt
%\usepackage[hang,bf]{caption2}%
%\usepackage{subfigure}%A
%\renewcommand{\subfigcapskip}{0pt}%
%\renewcommand{\subfigbottomskip}{-10pt}%
%\usepackage[T1]{fontenc}%
\usepackage{textcomp}                           % Enth\"{a}lt Spezialzeichen, z.B. Promille-Zeichen
\renewcommand{\rmdefault}{ptm}%                 % \rmdefault: Adobe Times Roman
%\usepackage{mathptmx}%
%\renewcommand{\mathbf}[1]{\mbox{\rmfamily\itshape\bfseries#1}}%
\usepackage{courier}%                           % \ttdefault: Adobe Courier
\usepackage[scaled=.92]{helvet}                 % \sfdefault: Adobe Helvetica
%\usepackage{xspace,inputenc}
%\usepackage{mathptmx}%
%\usepackage{ae}
%================================================================







%================================================================

%\setcounter{tocdepth}{1} %Tiefe des Inhaltsverzeichnis

%======================================
%Gabi's Befehle aus der DA



%================================
%\usepackage{fancyhdr}
%\pagestyle{fancy}
% with this we ensure that the chapter and section
% headings are in lowercase.
%\renewcommand{\chaptermark}[1]{%
%\markboth{#1}{}}
%\renewcommand{\sectionmark}[1]{%
%\markright{\thesection\ #1}}
%\fancyhf{} % delete current header and footer
%\fancyhead[LE,RO]{\bfseries\thepage}
%\fancyhead[LO]{\bfseries\rightmark}
%\fancyhead[RE]{\bfseries\leftmark}
%\fancyfoot[LE,RO]{{\tt\tiny \copyright{} Michael Koller\hfill \VersionNumber, \Created}}
%\fancyfoot[LO]{ }
%\fancyfoot[RE]{ }
%\renewcommand{\headrulewidth}{0.5pt}
%\renewcommand{\footrulewidth}{0pt}
%\addtolength{\headheight}{0.5pt} % space for the rule
%\fancypagestyle{plain}{%
%\fancyhead{} % get rid of headers on plain pages
%\renewcommand{\headrulewidth}{0pt} % and the line
%}


\ifcopyright
\usepackage{fancyhdr}
\pagestyle{fancy}
% with this we ensure that the chapter and section
% headings are in lowercase.
\renewcommand{\chaptermark}[1]{%
\markboth{#1}{}}
\renewcommand{\sectionmark}[1]{%
\markright{\thesection\ #1}}
\fancyhf{} % delete current header and footer
\fancyhead[LE,RO]{\bfseries\thepage}
\fancyhead[LO]{\bfseries\rightmark}
\fancyhead[RE]{\bfseries\leftmark}
\fancyfoot[LE,RO]{{\mycopyright}}
\fancyfoot[LO]{ }
\fancyfoot[RE]{ }
\renewcommand{\headrulewidth}{0.5pt}
\renewcommand{\footrulewidth}{0pt}
\addtolength{\headheight}{0.5pt} % space for the rule
\fancypagestyle{plain}{%
\fancyhead{} % get rid of headers on plain pages
\renewcommand{\headrulewidth}{0pt} % and the line
}
\fi

%================================================================
\newif\ifstandalloneKelf
\standalloneKelffalse

\newtheorem{defn}[conjecture]{Definition}{\bf}{\it}
%\newtheorem{example}[conjecture]{Beispiel}{\bf}{\rm}
\newtheorem{bsp}[conjecture]{Example}{\bf}{\rm}
%\newtheorem{exercise}[conjecture]{\"Ubung}{\bf}{\rm}
\newtheorem{ueb}[conjecture]{Exercise}{\bf}{\rm}
%\newtheorem{lemma}[conjecture]{Lemma}{\bf}{\it}
%\newtheorem{note}[conjecture]{Bemerkung}{\bf}{\rm}
%\newtheorem{problem}[conjecture]{Fragestellung}{\bf}{\rm}
%\newtheorem{property}[conjecture]{Eigenschaft}{\bf}{\rm}
%\newtheorem{proposition}[conjecture]{Satz}{\bf}{\it}
\newtheorem{satz}[conjecture]{Proposition}{\bf}{\it}
%\newtheorem{question}[conjecture]{Frage}{\bf}{\rm}
%\newtheorem{solution}[conjecture]{L�sung}{\bf}{\rm}
%\newtheorem{theorem}[conjecture]{Theorem}{\bf}{\it}
\newtheorem{thm}[conjecture]{Theorem}{\bf}{\it}
%\newtheorem*{proof}{Beweis}{\it}{\rm}
%\newtheorem{remark}[conjecture]{Bemerkung}{\bf}{\rm}
\newtheorem{bem}[conjecture]{Remark}{\bf}{\rm}
\newtheorem{todo}{TODO}[section]{\bf}{\rm}
