%================================================================
\usepackage{springer}
%\usepackage[german]{babel}%
%\usepackage{ae}%
\usepackage[T1]{fontenc}%
\usepackage[ansinew]{inputenc,}%
\usepackage{mksty} 
%\usepackage[latin1]{inputenc}
\usepackage{a4}%
\usepackage{makeidx,multind} % ,mathcomp}
\usepackage{enumerate}%
\usepackage{color}%
\selectlanguage{\english}
%\usepackage{amsfonts,amsmath,amssymb,amsthm} %% txfonts hinzugef\"{u}gt 2.12.2002
\usepackage{amsfonts} %% txfonts hinzugef\"{u}gt 2.12.2002
%\usepackage{amssymbb} %% txfonts hinzugef\"{u}gt 2.12.2002

%================================================================
%% Gesetzt von A. Wyler 2.12.2002
\setcounter{secnumdepth}{2}%
%\arraycolsep2pt
%\usepackage[hang,bf]{caption2}%
%\usepackage{subfigure}%A
%\renewcommand{\subfigcapskip}{0pt}%
\newcommand{\coloneqq}{:=}%
\newcommand{\Diamondblack}{\diamond}%
%\renewcommand{\subfigbottomskip}{-10pt}%
%\usepackage[T1]{fontenc}%
\usepackage{textcomp}                           % Enth\"{a}lt Spezialzeichen, z.B. Promille-Zeichen
\renewcommand{\rmdefault}{ptm}%                 % \rmdefault: Adobe Times Roman
%\usepackage{mathptmx}%
%\renewcommand{\mathbf}[1]{\mbox{\rmfamily\itshape\bfseries#1}}%
\usepackage{courier}%                           % \ttdefault: Adobe Courier
\usepackage[scaled=.92]{helvet}                 % \sfdefault: Adobe Helvetica
\usepackage[pdftex]{graphicx}

%\usepackage{ae}
 \usepackage{xspace,inputenc}
\usepackage{mathptmx}%
\usepackage{lscape}
\usepackage[pdftex,
            colorlinks=true,            % Schrift von Links in Farbe (true), sonst mit Rahmen (false)
            bookmarksnumbered=true,     % Lesezeichen im pdf mit Nummerierung
            bookmarksopen=true,         % �ffnet die Lesezeichen vom pdf beim Start
            bookmarksopenlevel=0,       % Default ist maxdim
            pdfstartview=FitH,          % startet mit Seitenbreite
            linkcolor=blue,             % Standard 'red'
            citecolor=blue,             % Standard 'green'
            urlcolor=blue,              % Standard 'cyan'
            filecolor=blue,             %
            pdfauthor = {\Authors },
            pdftitle ={\DocTitle -- Version \VersionNumber},
            plainpages=false,pdfpagelabels]{hyperref} %

\newfont{\ssmall}{phvr8t scaled 250}

\ifspringersize%
    \typeout{Kapitel Seitengr\"{o}sse: Springer}%
\else%
    \typeout{Kapitel Seitengr\"{o}sse: A4 Normal}%
    \textwidth=16cm%
    \textheight=23cm%
    \oddsidemargin=0.0cm%
    \evensidemargin=0.0cm%
\fi%
\typeout{================================================================}%
%================================================================

%\setcounter{tocdepth}{1} %Tiefe des Inhaltsverzeichnis

%======================================
%Gabi's Befehle aus der DA


\newcommand{\gl}{\qquad\Leftrightarrow\qquad}
\newcommand{\gk}{\quad\Leftrightarrow\quad}
\newcommand{\rg}{\qquad\Rightarrow\qquad}
\newcommand{\rk}{\Rightarrow\quad}
\newcommand{\h}{\hspace{5cm}}

\newcommand{\al}[1]{\begin{align*}{#1}\end{align*}}
\newcommand{\alb}[1]{\begin{align*}{#1}\end{align*}\bigskip}
\newcommand{\tab}[2]{\begin{center}\begin{tabular}{#1}{#2}\end{tabular}\end{center}}

\newcommand{\tabu}[1]{\hspace{-0.15cm}\begin{tabular}{ll}{#1}\end{tabular}}
\newcommand{\hta}{\hspace{-0.235cm}}
\newcommand{\ZCB}[1]{\mathcal{Z}_{(#1)}}
%================================
%\usepackage{fancyhdr}
%\pagestyle{fancy}
% with this we ensure that the chapter and section
% headings are in lowercase.
%\renewcommand{\chaptermark}[1]{%
%\markboth{#1}{}}
%\renewcommand{\sectionmark}[1]{%
%\markright{\thesection\ #1}}
%\fancyhf{} % delete current header and footer
%\fancyhead[LE,RO]{\bfseries\thepage}
%\fancyhead[LO]{\bfseries\rightmark}
%\fancyhead[RE]{\bfseries\leftmark}
%\fancyfoot[LE,RO]{{\tt\tiny \copyright{} Michael Koller\hfill \VersionNumber, \Created}}
%\fancyfoot[LO]{ }
%\fancyfoot[RE]{ }
%\renewcommand{\headrulewidth}{0.5pt}
%\renewcommand{\footrulewidth}{0pt}
%\addtolength{\headheight}{0.5pt} % space for the rule
%\fancypagestyle{plain}{%
%\fancyhead{} % get rid of headers on plain pages
%\renewcommand{\headrulewidth}{0pt} % and the line
%}

\parindent 0mm
\parskip 1.2ex plus 0.2ex minus 0.2ex
\parskip \baselinestretch\parskip

\newtheorem{theorem}{Theorem}[section]
\newtheorem{satz}[theorem]{Satz}
\newtheorem{lemma}[theorem]{Lemma}
\newtheorem{thm}[theorem]{Theorem}
\newtheorem{defn}[theorem]{Definition}
\newtheorem{definition}[theorem]{Definition}
\newtheorem{bsp}[theorem]{Beispiel}
\newtheorem{example}[theorem]{Beispiel}
\newtheorem{konv}[theorem]{Konvention}
\newtheorem{bem}[theorem]{Bemerkung}
\newtheorem{remark}[theorem]{Bemerkung}
\newtheorem{notation}[theorem]{Notation}
\newtheorem{ueb}{\"Ubung}
\newtheorem{todo}{TODO}

\newenvironment{proof}[0]{{\sc Beweis: \newline}}{\par \hfill $\Box$\par\bigskip}

\newcommand{\mb}[1]{\mbox{#1}}
\newcommand{\XT}[0]{(X_t)_{t\in T}}
\newcommand{\OAP}[0]{(\Omega,\mathcal{A},P)}
\newcommand{\PIJ}[2]{p_{ij}{(#1,#2)}}
\newcommand{\BEEQS}[0]{\begin{eqnarray*}}
\newcommand{\EEEQS}[0]{\end{eqnarray*}}
\newcommand{\BEEQ}[0]{\begin{eqnarray}}
\newcommand{\EEEQ}[0]{\end{eqnarray}}
\newcommand{\TODO}[1]{\begin{todo}[{\bf TO be done}] #1  \end{todo} }
\newcommand{\ABST}[0]{\mbox{\hspace*{1cm}}}
%\newcommand{\ZCB}[1]{\mathcal{Z}_{(#1)}}
\newcommand{\PRICE}[2]{\pi_{#1}\left({#2}\right)}
