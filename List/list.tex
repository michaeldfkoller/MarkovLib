\documentclass[11pt,a4paper]{article}%
%% ------------------------------------
\input{misc/packages}%
%% Spezielle Pakete f�r pdfLaTeX mit hypperref
\usepackage[T1]{fontenc}%
\usepackage[ansinew]{inputenc}%
\usepackage{a4}%
\usepackage{makeidx,multind,mathcomp}
\usepackage{enumerate}%
\usepackage{color}%
\usepackage{amsfonts,amsmath,amssymb,amsthm,txfonts} %% txfonts hinzugef\"{u}gt 2.12.2002
\usepackage{supertabular}%
\usepackage[pdftex]{graphicx}
\usepackage[pdftex,
            colorlinks=true,            % Schrift von Links in Farbe (true), sonst mit Rahmen (false)
            bookmarksnumbered=true,     % Lesezeichen im pdf mit Nummerierung
            bookmarksopen=true,         % �ffnet die Lesezeichen vom pdf beim Start
            bookmarksopenlevel=0,       % Default ist maxdim
            pdfstartview=FitH,          % startet mit Seitenbreite
            linkcolor=blue,             % Standard 'red'
            citecolor=blue,             % Standard 'green'
            urlcolor=blue,              % Standard 'cyan'
            filecolor=blue,             %
            plainpages=false]{hyperref} % Forces page anchors to be named by the arabic form of the page number,
                                        % rather than the formatted form. (default=true; Probleme wenn Seitennummer
                                        % im Dokument zur�ckgesetzt wird!)
            %pdfpagemode=UseThumbs      % Thumbnails
            %pdfhighlight=/O

\begin{document}%
\chapter{Application of the Markov model to Life Insurance} \label{App*Markov}
\typeout{****************************************************************}
\typeout{This is Version 0.80 of markov.tex}
\typeout{****************************************************************}
\section{Traditional Rating of Life Contracts}

Before starting with the Markov model, I would like to summarise how traditional 
calculations using commutation functions are performed. Usually one starts with the 
probabilities of death and then calculates a decrement table starting with, say, 100000 
persons at age 20. 

After that one, has to calculate the different commutation functions, which I assume 
everybody knows by heart. These numbers depend on the persons alive and 
on the technical interest rate i. Only when you have done this it is (in the 
classical framework) possible to calculate the necessary premiums. 
In the following we will look a little bit closer at the calculation of a single premium for an 
annuity. To do this we need the following commutation functions: 

\begin{eqnarray*} 
D_x &=& v \times l_x \mbox{where $l_x$ denotes the number of persons alive at age $x$.} \nonumber \\
C_x &=& v \times (l_{x+1} -l_x )  \nonumber
\end{eqnarray*}

Having this formalism it is well known that 

\begin{eqnarray*}
\ddot{a}_x & = \frac{N_x}{D_x}  
\end{eqnarray*}



From this example is easily seen that almost all premiums can be calculated by summation 
and multiplication of commutation functions. Such an approach has its advantages 
in an environment where calculations have to be performed by hand, or where computers are 
expensive. Calculation becomes messy if benefits are considered with 
guarantees or with refunds. 

The Markov model here presented offers rating of life contracts without using commutation 
functions. It starts with calculation of the reserves and uses the involved probabilities directly. 
In order to see such a calculation let's review the above-mentioned example: 
We will use ${}_n p_x$ to denote the probability of a person aged exactly $x$ surviving for $n$ years. 

\begin{eqnarray*}
\ddot{a}_x  & = & \sum_{j=0}^\infty {}_{j} p_x \times v^j \\ 
            & = & 1 + p_x \times \ddot{a}_{x+1}
\end{eqnarray*}

The above formula gives us a recursion for the mathematical reserves of the contract. Hence 
one can calculate the necessary single premiums just by recursion. In order to do this, we need 
an initial condition, which is in our case $V_\omega = 0$. 

The interpretation of the formula is easy: The necessary reserve at age $x$ consists of two parts: 
\begin{enumerate}
\item The annuity payment, and 
\item The necessary reserve at age x+1. (These reserves must naturally be 
discounted.) 
\end{enumerate}

It should be pointed out that the calculation does not need any of the commutation functions; 
only $p_x$ and the discount factor v are used. As a consequence this method does not produce 
the overheads of traditional methods. 

In the following paragraphs the discrete time, discrete state Markov model is introduced and 
solutions of some concrete problems are offered. 

At this point, it is necessary to stress the fact that the following frame work can be used, with 
some modifications, in an environment with stochastic interest. But as we are limited in space 
and time we have to restrict ourselves to deterministic constant discount rates. 

\section{Life Insurance considered as Random Cash flows}

The starting point of the Markov model is a set of states, which correspond to the different 
possible conditions of the insured persons. In life insurance the set of states usually consists of 
{alive, dead}. The set of states will be denoted by $S$. 

The second point which originates from the life contract has to do with the so-called 
contractual functions which depend on the states and the time. Hence the structure of a 
generalised life contract can be thought of: 

Contractual situation between time $t$ and time $t+1$ 

\mkimg{0.95}{img/markov.png}%

From the above diagram it can be seen that a finite number of states is considered, and that for 


each transition $i \to j$ two different sums are paid, namely $a_{ij}^{\mb{Post}}(t)$ at the end of the 
considered time interval and $a_i^{\mb{Pre}}(t)$ at the beginning of it. It is clear that the value of the 
payment stream $a_{ij}^{\mb{Post}}(t)$ has to be discounted by v in order to be compatible with $a_i^{\mb{Pre}}(t)$. 
Probably it is worth remarking that the use of the two payment streams $a_i^{\mb{Pre}}(t)$ and $a_{ij}^{\mb{Post}}(t)$ 
eases the solution of things like payments during the year and the distinction between lump 

sums (generally payable at the end of the period) and annuities (at the beginning). Finally it 
must be said that premiums payable to the insurer can (not must (!)) be considered as benefits 
with the opposite sign. 

Until now we have defined the sums which are payable if a certain insured event occurs. Now 
there has to be a probability law in order to rate the different transitions. In the following we 
denote by $p_{ij}(t,t+1)$ the probability of transition at time t from state $i \to j$ . Hence in the 
language of the above diagram there is one transition probability assigned to each line 
between two states. 

So summarising a Markov life insurance model consists of the following: \par
\begin{tabular}{lp{5.5cm}}
$S$ & A finite state space (set). \\
$((p_{ij}(t))_{(i,j) \in S^2})_{t \in (1,2, \dots \omega)}$ & The transition probabilities describing the Markov chain $X_t$ on $S$. \\
$((a_i^{\mb{Pre}}(t))_{i \in S})_{t \in (1,2, \dots \omega)} $ & The prenumerando benefits relating, paying at the beginning of the corresponding period. \\
$((a_{ij}^{\mb{Post}}(t))_{(i,j) \in S^2})_{t \in (1,2, \dots \omega)} $ & The postnumerando benefits relating, paying at the end of the corresponding period, if a transition $i \to j$ happens.\\
$((v_i(t))_{i \in S})_{t \in (1,2, \dots \omega)} $ & The yearly discount rate from $[t, t+1[$. We have $v_t=\sum_{j \in S} I_j(t) \, v_i(t)$. 
\end{tabular}

\section{Reserves, Recursion and Premiums}

One of the most important quantities in actuarial science is the prospective reserve, as the 
insurer must have this amount of money for each policy. Therefore the concept of the 
prospective reserve is known to all actuaries. It is defined to be the present value of the future 
cash flow $A$ given the information at present. Formally we write 

\[ V^+_j(t,A) := E[V(t, A \times \chi_{]t,\infty]})\, | \, X_t=j], \]


(where $j$ denotes the state at time t). This notation tells us, that the reserve depends heavily on the state 
of the policy. 

In the context of the above we have

\begin{eqnarray*}
\Delta A(t)   & = & \sum_{j \in S} I_j(t) \times a_i^{\mb{Pre}}(t) + \sum_{(i,j) \in S \times S} \Delta N_{ij}(t) \times a_{ij}^{\mb{Pre}}(t), \\
A(t)          & = & \sum_{k \le t} \Delta A(k), \\
\Delta V(t,A) & = & v(t) \Delta A(t), \\ 
              & = & v(t) \left[ \sum_{j \in S} I_j(t) \times a_i^{\mb{Pre}}(t) + \sum_{(i,j) \in S \times S} \Delta N_{ij}(t) \times a_{ij}^{\mb{Pre}}(t) \right], \\
v(t)          & = & \prod_{\tau \le t} \left[ \sum_{j \in S} I_j(\tau) \times v_j(\tau) \right]. 
\end{eqnarray*}
 
The direct calculation of the necessary reserves for the different states is not too easy if you 
consider a general time continuous Markov model. An advantage of this model is the 
existence of a powerful backwards recursion. 
The following formula (Thiele difference equation) allows the recursive calculation of the necessary reserves and hence of 
the necessary single premiums: 

\begin{eqnarray}
V_i^+(t) & = & a_i^{Pre}(t) + \sum_{j\in S}  v_i(t) \, \mbox{$p$}_{ij}(t) \big\{ a^{Post}_{ij}(t) + V_j^+(t+1) \big\}.
\end{eqnarray} \label{DK:Markov:Prop:Calc}

The interpretation of the formula is almost the same as in the trivial example at the beginning. In 
principle the present reserve consists of payments due to the different possible transitions and 
the discounted values of the future necessary reserves. It can be seen that the above recursion 
uses only the different benefits, the probabilities and the discount factor. 
In order to calculate the reserve for a certain age one has to do a backwards recursion starting 
at the expiration date of the policy. For annuities this is usually the age $\omega$ when everybody 
has died. Starting the recursion it is necessary to have boundary conditions, which depend on 
the payment stream at the expiration date. Usually the boundary conditions are taken to be 
zero for all reserves. It should be pointed out that one has to do this recursion for the reserves 
of all states simultaneously. 

After the calculation of the different reserves one can naturally determine the corresponding 
necessary single premiums by the principle of equivalence. 

We want to end this section with a short proof of the above mentioned Thiele recursion:

We know that $A(t) = \sum_{k \le t} \Delta A(k) $ and also that 

\[
\Delta V(t,A) = v(t) \left[ \sum_{j \in S} I_j(t) \times a_i^{\mb{Pre}}(t) + \sum_{(i,j) \in S \times S} \Delta N_{ij}(t) \times a_{ij}^{\mb{Pre}}(t) \right]. 
\]

Hence we have

\begin{eqnarray*}
V_i^+(t) & = & \frac{1}{v(t)} \mathbb{E} \left[ \sum_{\tau = t}^ \infty v(\tau) \times \Delta A(\tau) \, | \, X_t=i \right] \\
         & = & \frac{1}{v(t)} \mathbb{E} \left[ \sum_{j \in S} I_j(t+1) \times \sum_{\tau = t}^ \infty v(\tau) \times \Delta A(\tau) \, | \, X_t=i \right] ,
\end{eqnarray*}

remarking that $\sum_{j \in S} I_j(t+1) = 1$. If we now consider all the terms in $\Delta A(t)$ for a given $I_j(t+1)$ for $j \in S$, it becomes obvious that the Markov chain changes from $i \to j$ and in consequence only $N_{ik}(t)$ increases by one for $k=j$. If we furthermore use the projection property and the linearity of the conditional expected value and the fact that $\mathbb{E}\left[I_j(t+1) \, | \, X_t = i \right] = p_{ij}(t,t+1)$, together with the Markov property, we get the formula if we split $V_i^+(t)$ as follows:

\begin{eqnarray*}
V_i^+(t) & = & \frac{1}{v(t)} \mathbb{E} \left[ \sum_{\tau = t}^ \infty v(\tau) \times \Delta A(\tau) \, | \, X_t=i \right] \\
         & = & \frac{1}{v(t)} \mathbb{E} \left[ \left\{ \sum_{\tau = t}^t + \sum_{\tau = t+1}^ \infty\right\}v(\tau) \times \Delta A(\tau) \, | \, X_t=i \right]. 
\end{eqnarray*}


Doing this decomposition we get for the first part:

\begin{eqnarray*}
\mb{Part}_1 & = & a_i^{Pre}(t) + \sum_{j\in S}  v_i(t) \, \mbox{$p$}_{ij}(t) a^{Post}_{ij}(t),
\end{eqnarray*}


and for the second:

\begin{eqnarray*}
\mb{Part}_2 & = & \sum_{j\in S}  v_i(t) \, \mbox{$p$}_{ij}(t) V_j^+(t+1).
\end{eqnarray*}

Adding the two parts together we get the desired result:


\begin{eqnarray*}
V_i^+(t) & = & a_i^{Pre}(t) + \sum_{j\in S}  v_i(t) \, \mbox{$p$}_{ij}(t) \big\{ a^{Post}_{ij}(t) + V_j^+(t+1) \big\}.
\end{eqnarray*}

More concretely we have

\begin{eqnarray*}
V_i^+(t)  & = & \frac{1}{v(t)} \mathbb{E} \left[ \sum_{j \in S} I_j(t+1) \times \sum_{\tau = t}^ \infty v(\tau) \times \Delta A(\tau) \, | \, X_t=i \right] \\
          & = & a_i^{Pre}(t) + \sum_{j \in S} \mathbb{E} \left[ I_j(t+1) \times \sum_{\tau = t}^ \infty \frac{v(\tau)}{v(t)} \times \Delta A(\tau) \, | \, X_t=i \right] \\
          & = & a_i^{Pre}(t) + \sum_{j \in S} \mathbb{E} \bigg[ I_j(t+1) v_i(t) \bigg\{ a_{ij}^{Post} +  \nonumber \\ 
          & & +   \mathbb{E} \bigg[ \sum_{\tau = t+1}^ \infty \frac{v(\tau)}{v(t+1)} \times \Delta A(\tau) \, \bigg| \, X_t=i, X_{t+1}=j  \bigg] \bigg\} \bigg| \, X_t=i \bigg] \\
          & = & a_i^{Pre}(t) + \sum_{j\in S}  v_i(t) \, \mbox{$p$}_{ij}(t) \big\{ a^{Post}_{ij}(t) + V_j^+(t+1) \big\}.
\end{eqnarray*}


We remark that this section can only be a short introduction to this topic and we refer to \cite{Koller2010} for a more extensive discussion.

\section{Beispiele}
\section{omarkov.h}
\section{omarkov.cpp}
\section{annuity.h}
\section{annuity.cpp}
\section{capital.h}
\section{capital.cpp}
\end{document}%
