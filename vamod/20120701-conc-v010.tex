
\RequirePackage{lineno}
\documentclass[10pt, a4paper,landscape]{article}
% -----------------------------------------------------*
\newif\ifmktable \mktablefalse
\newif\ifmkcomments \mkcommentsfalse
\newif\ifPrincipal \Principaltrue
\usepackage{multirow}
\usepackage{enumitem}
\newenvironment{mytabular}{\begin{footnotesize} \begin{center}}{\end{center} \end{footnotesize} }
\newenvironment{mytabularleft}{\begin{footnotesize} \begin{flushleft}}{\end{flushleft} \end{footnotesize} }
\newcommand{\MKcomment}[2]{\marginpar{{\underline{#1:}}\\{#2}}}
\newcommand{\MKK}[1]{\footnote{$\kappa$: #1}}
\newcommand{\MKKM}[1]{\ifmkcomments \MKcomment{Michael}{\textred{#1}} \fi}
\newcommand{\DocData}{\today}%
\newcommand{\DocTitle}{Concept VA Policy Calculation}
\newcommand{\ReportNo}{\jobname - \today}
\newcommand{\vega}[0]{\nu}
\newcommand{\myindex}[1]{#1\index{#1}}
\newcommand{\mybox}[3]{\makebox[#1\textwidth][#2]{#3}}

\newcommand{\mkeinbildxx}[2]{
      \begin{center}
      \setlength{\fboxrule}{0.0075\textwidth}\fcolorbox{mygrey}{mylightgrey}{
      \includegraphics[width=#1\textwidth]{#2}
      }
      \end{center}
       }

%\begin{center}\includegraphics[width=#1\textwidth]{#2}\end{center}}

\newcommand{\mkeinbildbigxx}[4]{\vfill
       \begin{center}%
       \setlength{\fboxrule}{#4\textwidth}\fcolorbox{mygrey}{mylightgrey}{
       \includegraphics[angle=#1,height=#2 \textheight]{#3} 
       }
       \end{center} \vfill
       }
%\begin{center}\includegraphics[width=#1\textwidth]{#2}\end{center}}

\newcommand{\mkeinbild}[1]{                    
                    \mkeinbildxx{0.95}{#1}
                    %\vspace*{2mm}
      }
\newcommand{\mkeinbildmod}[2]{\begin{center}                    
                    \mkeinbildxx{#1}{#2} 
      %\vspace*{2mm}
      \end{center}
      }

\newcommand{\mkzweibild}[2]
{                   \begin{center}
                    \begin{minipage}{.495\textwidth}%
                    \mkeinbildxx{0.95}{#1}
                    \end{minipage}\hfill%
                    \begin{minipage}{.495\textwidth}%
                    \mkeinbildxx{0.95}{#2}
                    \end{minipage}%
%\vspace*{2mm}
                  \end{center}
               }

\newcommand{\mkimg}[2]{
\includegraphics[width=0.95\textwidth]{#2}}

\newcommand{\locpic}[1]{
\includegraphics[width=0.90\textwidth]{#1}
}

%----------------------------------------------------------
\usepackage[pdftex]{graphicx}
\usepackage{a4}%
\usepackage{amssymb}%
\usepackage{courier}%                   % \ttdefault: Adobe Courier
\usepackage{color}%
\definecolor{mygrey}{rgb}{0.7,0.7,0.7}
\definecolor{mylightgrey}{rgb}{0.7,0.7,0.7}
\usepackage[pdftex]{graphicx}
\usepackage[scaled=.92]{helvet}         % \sfdefault: Adobe Helvetica
\renewcommand\familydefault{phv}
\ifmkcomments
\message{USE LARGER PAPER}
\setlength{\paperwidth}{240mm}
\else
\setlength{\paperwidth}{210mm}
\fi
\setlength{\paperheight}{297mm}%
    \textwidth=16cm% was 16cm in order to accomodate margin
    \textheight=23cm%
    \oddsidemargin=0.0cm%
    \evensidemargin=0.0cm%
\parindent=0mm

\definecolor{myred}{rgb}{1.0,0,0}
\definecolor{myblue}{rgb}{0,0,0.9}
\definecolor{mygreen}{rgb}{0.27,0.52,0.21} % 26/90/23 68/135/55
\newcommand{\textred}[1]{\textcolor{myred}{#1}}                    
\newcommand{\textblue}[1]{\textcolor{myblue}{#1}}                    
\newcommand{\textgreen}[1]{\textcolor{mygreen}{#1}}                    
\usepackage{fancyhdr}
\usepackage{lscape}
\pagestyle{fancy}
\fancyhf{} % delete current header and footer
%\fancyhead[LE,RO]{\bfseries\thepage}
%\fancyhead[LO]{{\sc \DocTitle: \DocData} }
%\fancyhead[RE]{\bfseries\leftmark}
%\fancyfoot[LE,RO]{Confidential - No onwards sharing.}
%\fancyfoot[LO]{{\sc This document is confidential and contains price sensitive information}}
%\fancyfoot[RE]{{\sc This document is confidential and contains price sensitive information}}
\fancyfoot[RE,RO]{\thepage}
\fancyhead[RE,RO]{{\large \textred{CONFIDENTIAL}}}
\fancyhead[LE,LO]{{\large \sc \textred{\DocTitle}}}
\fancyfoot[LE,LO]{{\tiny \ReportNo}}
\renewcommand{\headrulewidth}{0pt}
\addtolength{\headheight}{1.5pt}

\typeout{PDF Latex Support}
 \usepackage[pdftex,
            colorlinks=true,            % Schrift von Links in Farbe (true), sonst mit Rahmen (false)
            bookmarksnumbered=true,     % Lesezeichen im pdf mit Nummerierung
            bookmarksopen=true,         % Öffnet die Lesezeichen vom pdf beim Start
            bookmarksopenlevel=0,       % Default ist maxdim
            pdfstartview=FitH,          % startet mit Seitenbreite
            linkcolor=blue,             % Standard 'red'
            citecolor=blue,             % Standard 'green'
            urlcolor=blue,              % Standard 'cyan'
            filecolor=blue,             %
            pdfauthor = {Michael\ Koller},
            pdftitle ={\DocTitle: \DocData},
            plainpages=false,pdfpagelabels]{hyperref} %
\usepackage[final]{pdfpages}
\usepackage{longtable}
\pagestyle{empty}
\begin{document}
% ================================================================
\begin{flushleft}
Michael Koller \hfill Herrliberg, \today \\
Feldstrasse 14 \\
CH 8704 Herrliberg \\
Phone: +41 44 915 41 47\\
Switzerland \\[1cm]
\end{flushleft}

\tableofcontents

\section{Aim}
The aim of {\tt vamod} is to provide a library, where for a a given protfolio of policies and a given performance vector the respective cash flows for VA products can be calculated.

The driver routine, simulating the different preformace vectors is out of scope. The respective functunality is partially provided by {\tt simlib}. This module is intended to be integrated in the {\tt MARKOVLV} family

\section{Products Modelled}
There are the following VAs:
\begin{description}
\item[GMDB:] This the VA analogon of a term insurance. Hence the loss in case of death is the difference between the agreed contractual death benefit and the respective funds value.
\item[GMAB:] This the VA analogon of a pure endowment insurance. Hence the loss in case of surviving the term of the insurance is the difference between the agreed contractual death benefit and the respective funds value.
\item[GMIB:] This is roughly speakting a GMIB where at maturity the respective guarantee is converted in an immediate payout annuity. This product will not be modelled in {\tt vamod}
\item[GMWB:] This the VA analogon of an immediate or deferred annuity. Hence the loss are the annuityies which can not be taken out of the account value
\end{description}

\section{Calculation Formulae}

{\bf General Notation}
In the following section I summarise my understanding of the economics of this sort of contract. Assume the following:
\begin{itemize}
\item Person aged $x_0$ purchases such a GMWB and pays a single premium $EE$;
\item Assume that the person starts to withdraw at age $sw$ and that the income phase starts at age $s$;
\item We use the following notation:
\subitem $F(t)$: Funds value at time $t$. To be more precise we denote with $F(t)^-$ and $F(t)^+$ the value of the funds before and after withdrawal of the annuity, respectively. In consequence we have $F(t)^+ =F(t)^- - R(t)$.
\subitem $GWB(t)$: GWB (``Guaranteed Withdrawal Balance'') value at time $t$. To be more precise we denote with $GWB(t)^-$ and $GWB(t)^+$ the value of the funds before and after withdrawal of the annuity, respectively. In consequence we have $GWB(t)^+ =GWB(t)^- - R(t)$.
\subitem $f(x)$: GAWA percentage if person starts to withdraw at age $x$;
\subitem $GAWA(t)$: maximal allowable withdrawal benefit (usually $=GWB \times f(x)$).
\subitem $R(t)$: actual amount withdrawn. Note that we have the following: $R(\xi) = 0$ for $\xi < sw$, $ 0 \le R(\xi) \le GAWA(x)$, for all $\xi \in [sw, s[$, and  $R(\xi) = GAWA(x)$, for all $\xi \ge s$, assuming that the ``for life option'' is in place and identifying $x$ and $t$ in the obvious way, eg $x(t) = t-t_0 + x_0$  
\item With $\eta(t,\tau) \in \mathbb{R}$, we denote the fund performance during the time interval $[t,\tau]$, with $\tau > t$.
\item We do not allow for changes in funds and lapses at this time and also do not consider the death of the person insured. This would add some complexity, where actually the annuities need to be weighted with the respective probabilities ${}_tp_{x}$ and in the same sense the respective death cover weighted with ${}_tp_x \, q_x$. For the moment assume that $\sum{\tau \ge 0}$ stands for ``until death''. 
\item By $X(t)$ we denote the loss at time $t$ occurring from GMWB guarantees. It is obvious that under these premises the value of the total guarantee $Y = \sum_{\tau \ge 0} (1+r(\tau))^{-\tau} X(\tau) $, where $r(\tau)$ represents the risk free interest between $[0,\tau]$.
\end{itemize}

As described above we have the following:

\begin{eqnarray*}
f(x) &=& \left\{ 
\begin{array}{cc}
5 \% & \mbox{ if } x \in [55,74], \\
6 \% & \mbox{ if } x \in [75,84], \\
7 \% & \mbox{ if } x \ge 85.
\end{array} \right.
\end{eqnarray*}

For the recursion we have at time $t_0=0$:
\begin{eqnarray*}
F(0)   & = & EE, \\
GWB(0)   & = & EE, \\
GAWA(0)& = & f(sw) \times GWB(0) .
\end{eqnarray*}

Afterwards from time $t-1 \leadsto t$ we have the following:
\begin{eqnarray*}
F(t)^- &= & (1 + \eta(t-1,t)) \times F(t-1)^+, \\
GWB(t)^- & = & \max\{ GWB(t-1)^+,\max_{k=0,1,\dots, 4} \{1 + \eta(t-1,t-1 + \frac{k}{4})  \} \times F(t-1)^+\},\\
GAWA(t) & = & \max(GAWA(t-1), f(sw) \times GWB(t)^-) ,\\ 
F(t)^+ & = & \max(0, F(t)^- - R(t)), \\
GWB(t)^+ & = & \max(0, GWB(t)^- - R(t)), \\
X(t) & = & \max(0,R(t) - F(t)^-), \\[1ex] % Note shame on me: there as an error before namely F - R instead of R -F 
\pi(Y) & = & E^{Q}\left[\sum_{\tau \ge 0} (1+r(\tau))^{-\tau} X(\tau)  \right].
\end{eqnarray*}


If we now pick a given, mortality cover -- the simplest one -- namely the payment of $GWB(t)$ in case of death, we can calculate the value of the insurance option as follows:

\begin{eqnarray*}
\pi(Y) & = & E^Q\left[\sum\limits_{\tau \ge 0}^{\infty} (1+r(\tau))^{-\tau} X(\tau) \times {}_{\tau} p_{x_0} \right],
\end{eqnarray*}

{\bf Calculation of Funds and Losses}
As opposed to the specific setting, we aim to define the change in funds value more generally, namely:
\begin{eqnarray*}
T(\omega) & = & \mbox{Future life span. T=x: means person dies aged x}\\
F(t)      & = & \mbox{As above fund value at time t}\\
CF(t)     & = & \mbox{Cash flow at time t} \\
\eta(t-1,t)&= & \mbox{Funds performance} \\
F(t)^-    &= & (1 + \eta(t-1,t)) \times F(t-1)^+, \\
F(t)^+    & = & \max(0, F(t)^- - CF(t)), \\
X(t)      & = & \max(0,R(t) - F(t)^-), \\[1ex] % Note shame on me: there as an error before namely F - R instead of R -F  
\pi(Y) & = & E^{Q\times S}\left[\sum_{\tau \ge 0} (1+r(\tau))^{-\tau} X(\tau)  \right].
\end{eqnarray*}
Note that $S$ represents the probability measure with respect to the state the PHs are in. \par

{\bf Reference Quantity for benefits}

With $R(t)$ we denote the ratcheted up funds value if ratcheting is present. With $G(t)$ we denote the guarantee value. We have the following:

\begin{eqnarray*}
R(t+1) & = & \max(R(t), S(t)) \times \delta_{RA=1} + S(t) \times  \delta_{RA=0} \\
G^{exp}(t)   & = & \left\{
\begin{array}{ccc}
1 & \mbox{if} & x< x_0 \\
(1 + \alpha) ^{x-x_0} & \mbox{if} & x_0 \le x < x_1 \\
(1 + \alpha) ^{x1-x_0} & \mbox{else}
\end{array}
\right. \\
G^{lin}(t)   & = & \left\{
\begin{array}{ccc}
1 & \mbox{if} & x< x_0 \\
(1 + \alpha\times (x-x_0)) & \mbox{if} & x_0 \le x < x_1 \\
(1 + \alpha\times (x1-x_0)) & \mbox{else}
\end{array}
\right. \\
G(t0 & = & G^{lin} \times \delta_{lin = 1} + G^{exp} \times \delta_{exp = 1} \\
BE(t) & = & \max(S(t),R(t),G(t))\\
CF(t) & = & \sum_{i=1}^n BE(t) \times I_{Event\ i \ happens}(t) \times \beta_i
\end{eqnarray*}

Note that i represents the cases ``Death'', ``Maturity'', ``Annuity payment'', ``Premium Payment''. As example for $i=$ ``Death'' we have $I_{Death}(t) =\delta_{T=t}$.





\section{Structures}
\begin{verbatim}
typdef struct VABENEFITS
{
  // Definition of Guarantee Vector
  double dStartValueGuarantee; 
  double dIncreasePA;
  int    iStartGuaranteeAge;          // x_0
  int    iEndGuaranteeAge;            // x_1
  bool   bLinear;
  bool   bExponential;
  // Take also Fund Value into Account and make max - and how (eg Ratchet)
  bool   bMaxWithFunds; //Otherwise only guarantee
  int    iRatchet;  // 0 - no otherwise every iRatchet Periods
                                     // RA
  // Which Types of Benfits
  // note if Age < Current Age --> No Benefit
  int    iEndowmentAge; //0 - no endowment - otherwise maturity age
  int    iSTerm; // 0 - no term benefit otherwise s-age
  int    iSAnnuity; // Start age Annuity
  int    iSLastAnnuity; // Age at which Annuity ceases (\infty for lifelong)
  int    iSPrem; //Last Age with Premium 
  // Levels of Benefits -- these are the beta's
  double dPctEndowment;              // F(t)
  double dPctTerm;                   // R(t)
  double dPctAnnuity;                // CF(t)
  double dPctPremium;                // X(t)
} VABENEFITS;

typedef struct VAINVESTMENT
{
  // This Structure is also for rolling forwards
  double dEE;
  double dSAA[NRFUNDS];
  int iAgeRiskFree; // Means if Age >=iAgeRiskFree All assets in risk free (asset 0) 
  // This are the current Cash Flows and 
  double dPerformance[NRFUNDS];
  double dCurrentVA;
  double dCurrentRatchet;
  double dCurrentCashFlow;
  double dCurrentLoss;
} VAINVESTMENT;

typedef struct VAPERSON
{
  long lId
  int iAge;
  int iGender;
  int iBirthYear;
  VABENEFITS * psymB;
  VAINVESTMENT * psymI; 
} VAPERSON;
\end{verbatim}


% ================================================================
\end{document}
